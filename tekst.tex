\documentclass[a4paper,11pt]{article}
\usepackage[dutch]{babel}
\usepackage[utf8]{inputenc}
\usepackage{amsmath}
\usepackage{geometry}
\usepackage{parskip}
\usepackage{graphicx}
\usepackage{float}
\usepackage{xcolor} % Nodig voor kleuren

% Marges instellen
\geometry{a4paper, total={170mm,257mm}, left=20mm, top=20mm}

% --- DEFINITIE VAN HET ANTWOORD-BLOK ---
% Alles tussen \begin{antwoord} en \end{antwoord} wordt blauw.
\newenvironment{antwoord}
    {
    \vspace{0.1cm}      % Beetje witruimte boven
    \color{blue}        % Tekstkleur blauw
    \textbf{} \\ % Labeltje
    }
    {
    \vspace{0.2cm}      % Witruimte onder
    \hrule              % Horizontale lijn
    \vspace{0.3cm}      % Witruimte na de lijn
    \color{black}       % Terug naar zwart (veiligheid)
    }
% ---------------------------------------

\title{\textbf{Volledige Uitwerking Examenvragen}\\ \large Ingenieur en Economie (Versie 2025-2026)}
\author{Louic Cooreman}
\date{\today}

\begin{document}

\maketitle
\tableofcontents
\newpage

% =================================================================
\section{Module 1: Fundamenten van de Marketing}
% =================================================================

\subsection{Product Life Cycle (PLC) Figuur}
De onderstaande figuur stelt de PLC voor. Vul de fasen (1 t/m 5) aan en zet de assen erbij.

\begin{antwoord}
% TYP HIERONDER JE ANTWOORD
\begin{figure}[h] % [h] staat voor 'here', probeer de figuur hier te zetten
    \centering % Zet de foto in het midden van de pagina
    \includegraphics[width=0.6\textwidth]{PLC.png} % Pas de breedte aan
    \caption{Product Life Cycle} % Het bijschrift
    \label{fig:PLC} % Hiermee kun je in de tekst verwijzen naar de figuur
\end{figure}
\end{antwoord}


\subsection{Marketingdoelstellingen en Prijszetting in de PLC}
Bespreek kort de marketingdoelstelling en de prijszetting in elke fase van de PLC.

\begin{antwoord}
% TYP HIERONDER JE ANTWOORD
Er zijn verschillende marketingdoelstellingen , overleven , winstmaximalisatie , marktaandeel leiderschap , kwaliteit leiderschap. Tijdens de fase van de PLC worden deze allemaal toegepast. De product development stage laten we even vallen want hier wordt nog niks verkocht. Bij de introductie is het doel om bekendheid te creëren en klanten aan te trekken. De prijszetting kan een penetratieprijsstrategie zijn om snel marktaandeel te winnen of een afroomprijsstrategie om de vroege adopters te benutten. In de groeifase is het doel om marktaandeel te vergroten en de winst te maximaliseren. Prijzen kunnen stabiel blijven of licht dalen om concurrentie aan te gaan. Tijdens de volwassenheidsfase is het doel om marktaandeel te behouden en winstgevendheid te maximaliseren. Prijzen kunnen verder dalen als reactie op concurrentie. In de neergangsfase is het doel om kosten te minimaliseren en winst te behouden. Prijzen kunnen worden verlaagd om resterende vraag aan te trekken of producten uit de markt te halen (overleven).
\end{antwoord}


\subsection{Elasticiteit van de vraag}
Wat is elasticiteit van de vraag? Leg uit aan de hand van een schets en een voorbeeld.

\begin{antwoord}
% TYP HIERONDER JE ANTWOORD
Het toont aan hoe hard de de vraag van een product/dienst kan variëren met een variërende verkoopprijs. Als de verkoopprijs harder kan variëren spreken we over inelastisch gedrag , als de vraag harder variërd bij een lichtere verkoopprijs variatie dus dan spreken we over elastisch gedrag.
\begin{figure}[H] % Gebruik H om de figuur exact hier te plaatsen (vereist \usepackage{float})
    \centering % Zet de foto in het midden van de pagina
    \includegraphics[width=0.8\textwidth]{elsticitiet.png} % Pas de breedte aan
    \caption{Elasticiteit} % Het bijschrift
    \label{fig:Elasticiteit} % Hiermee kun je in de tekst verwijzen naar de figuur
\end{figure}
\end{antwoord}

\subsection{De 5 Marketingconcepten}
Geef de 5 marketingconcepten. Bespreek elk concept kort en geef een voorbeeld van elk.

\begin{antwoord}
% TYP HIERONDER JE ANTWOORD
\textbf{1. Productieconcept} : hierbij worden hoge productie snelheden gewenst aan een zo laag mogelijke prijs , typisch als de vraag hoger is dan aanbod , nadeel is dat er te hard gefocust wordt op productie , te weinig op klant. (voorbeeld Ford Model T)
\newline
\textbf{2. Productconcept} : hierbij ligt de focus op een product van hoge kwaliteit , typisch als gestreefd wordt naar het beste van het beste op de markt , nadeel is dat er te hard gefocust wordt op het product zelf en niet op de klant. (voorbeeld Apple iPhone)
\newline
\textbf{3. Verkoopconcept} : hierbij ligt de focus op zoveel mogelijk verkopen van je product door te promoten vaak , typisch als er meer aanbod dan vraag is , nadeel is dat er te hard gefocust wordt op verkopen en niet op klantbehoeften. (voorbeeld verzekeringsmaatschappijen)
\newline
\textbf{4. Marketingconcept} : hierbij ligt de focus op de klantbehoeften en wensen , typisch als er veel concurrentie is op de markt , nadeel is dat het tijdrovend en duur kan zijn om klantbehoeften te onderzoeken en het bedrijf hun eigen innovatie kan remmen. (voorbeeld IKEA)
\newline
\textbf{5. Maatschappelijk marketingconcept} : hierbij ligt de focus op het creëren van waarde voor klanten op een manier die ook goed is voor de samenleving als geheel , typisch als er veel aandacht is voor duurzaamheid en ethiek , nadeel is dat het moeilijk kan zijn om een balans te vinden tussen winstgevendheid en maatschappelijke verantwoordelijkheid. (voorbeeld Patagonia)
\end{antwoord}


\subsection{Het Marketingmanagementconcept}
Definieer 'marketingmanagementconcept'. Wat is het belang ervan voor consument en organisatie? Bespreek de 5 belangrijkste en geef duidelijk de verschilpunten aan.

\begin{antwoord}
% TYP HIERONDER JE ANTWOORD
Het marketingmanagementconcept is een benadering waarbij bedrijven zich richten op het identificeren en voldoen aan de behoeften en wensen van hun doelmarkt om zo hun doelen te bereiken. Het belang ervan ligt in het creëren van waarde voor zowel de consument als de organisatie, wat leidt tot klanttevredenheid, loyaliteit en uiteindelijk winstgevendheid voor het bedrijf. De 5 belangrijkste concepten zijn in de vraag hierboven besproken.
\end{antwoord}


\subsection{Segmentatie en Clustering}
\begin{itemize}
    \item Wat zijn segmentatievariabelen? Bespreek de belangrijkste categorieën en geef een voorbeeld van elk.
    \item Wat wordt bedoeld met 'clustering van doelgroepen'? Leg uit.
\end{itemize}

\begin{antwoord}
% TYP HIERONDER JE ANTWOORD
Bedrijven kunnen vaak niet iedereen als klant hebben , daarom wordt de maatschappij opgedeeld in segmentatievariabelen. Dit zijn eigelijk groepjes van mensen met gelijkwaardige kenmerken waarop gefocust kan worden om zo een product of dienst te verkopen. De belangrijkste categorieën zijn :
\newline
\textbf{1. Demografische variabelen} : leeftijd , geslacht , inkomen , opleiding , beroep.
\newline
\textbf{2. Geografische variabelen} : locatie , klimaat , bevolkingsdichtheid
\newline
\textbf{3. Psychografische variabelen} : levensstijl , persoonlijkheid , waarden
\newline
\textbf{4. Gedragsmatig} : koopgedrag , gebruiksfrequentie , merkentrouw
\newline
Clustering van doelgroepen is het proces waarbij bedrijven deze segmentatievariabelen gebruiken om specifieke groepen consumenten te identificeren die vergelijkbare behoeften en gedragingen vertonen. Door deze groepen te clusteren, kunnen bedrijven gerichte marketingstrategieën ontwikkelen die beter aansluiten bij de wensen van elke doelgroep, wat leidt tot effectievere communicatie en hogere conversieratio's.
\end{antwoord}


\subsection{Prijsstrategieën}
Wat wordt bedoeld met een \textbf{penetratieprijsstrategie} en een \textbf{afroomprijsstrategie}? Wanneer pas je ze toe?

\begin{antwoord}
% TYP HIERONDER JE ANTWOORD
De afstroomprijsstrategie is een strategie waarbij een bedrijf een product lanceert met een hoge prijs en deze later gaat dalen. Dit werkt goed bij bedrijven die al een groot imago hebben zodat hun trouwe klanten eerst kunnen kopen. De penetratiestrategie is een strategie waarbij een bedrijf een product lanceert met een lage prijs om snel marktaandeel te winnen. Dit werkt goed in markten met veel concurrentie of als het product nieuw is en klanten moet aantrekken.
\end{antwoord}


\subsection{Prijsvork}
Wat wordt bedoeld met een 'prijsvork'?

\begin{antwoord}
% TYP HIERONDER JE ANTWOORD
Het is eigenlijk een range van een minimale en een maximale prijs waarvoor een klant voor een product wilt betalen. Minimaal omdat anders de klant de kwaliteit niet vertrouwd , maximaal omdat dat het max is dat de klant wilt betalen. Vaak wordt deze prijsvork bepaald door marktonderzoek.
\end{antwoord}


\subsection{Integratievraag}
Ondernemerschap, Maatschappelijke ontwikkeling, Investeringen, Productontwikkeling, Toevoeging van waarde, Kostprijscalculatie, Gezonde bedrijven en welvaart... Hoe hangt het allemaal aan elkaar? Bespreek.

\begin{antwoord}
% TYP HIERONDER JE ANTWOORD
Wa een domme vraag? In 1 woord? \textbf{ECONOMIE}.
\newline
Hier is wat AI zegt :
\newline
Al deze concepten zijn met elkaar verbonden in het bredere kader van bedrijfsvoering en economische groei. Ondernemerschap stimuleert innovatie en productontwikkeling, wat leidt tot nieuwe producten en diensten die waarde toevoegen voor consumenten. Maatschappelijke ontwikkeling beïnvloedt de behoeften en verwachtingen van consumenten, wat bedrijven ertoe aanzet om hun producten en diensten aan te passen. Investeringen zijn cruciaal voor het financieren van deze innovaties en het uitbreiden van bedrijfsactiviteiten. Kostprijscalculatie helpt bedrijven om hun kosten te begrijpen en winstgevendheid te waarborgen, wat essentieel is voor het creëren van gezonde bedrijven. Gezonde bedrijven dragen bij aan economische welvaart door banen te creëren, belastingen te betalen en bij te dragen aan de gemeenschap. Kortom, deze elementen werken samen om een dynamische en duurzame economische omgeving te bevorderen.
\end{antwoord}


\subsection{De Marketingmix (4P's)}
Bespreek de 4P's in het kader van de 'Marketingmix'.

\begin{antwoord}
% TYP HIERONDER JE ANTWOORD
De 4P's van de marketingmix zijn Product, Prijs, Plaats en Promotie.
\newline
\textbf{Product} : Dit verwijst naar het fysieke product of de dienst die een bedrijf aanbiedt aan zijn klanten. Het omvat aspecten zoals ontwerp, kwaliteit, functies, merknaam en verpakking.
\textbf{Prijs} : Dit verwijst naar de hoeveelheid geld die klanten moeten betalen om het product of de dienst te verkrijgen. Prijsstrategieën kunnen variëren afhankelijk van factoren zoals concurrentie, kosten en klantperceptie van waarde.
\newline
\textbf{Plaats} : Dit verwijst naar de distributiekanalen en locaties waar het product of de dienst beschikbaar is voor klanten. Het omvat beslissingen over winkels, online verkoop, logistiek en voorraadbeheer.
\newline
\textbf{Promotie} : Dit verwijst naar de marketingcommunicatieactiviteiten die een bedrijf gebruikt om zijn product of dienst onder de aandacht van klanten te brengen. Dit kan reclame, verkoopbevordering, public relations en persoonlijke verkoop omvatten.
\end{antwoord}


\subsection{Prijszetting factoren}
Welke factoren zijn van belang voor de prijszetting? Splits ze op naar 'interne' en 'externe' factoren.

\begin{antwoord}
% TYP HIERONDER JE ANTWOORD
\textbf{Interne factoren} : marketingdoelstellingen , marketingstrategieën , kosten , prijszetting
\newline
-Die marketingdoelstellingen zijn die inspanningen dat een bedrijf kan doen die we bij de bespreking van de PLC zagen : overleven , winstmaximalisatie , marktaandeel leiderschap , kwaliteit leiderschap.
\newline
-Die marketingstrategieën zijn de 4P's van de marketingmix : product , prijs , plaats , promotie.
\newline
-Kosten zijn dingen zoals vaste kosten en variabele kosten. 
\newline
-Prijszetting met bvb : de prijsvork.
\newline
\textbf{Externe factoren} : de aard van de markt , vraag , concurrenten , economie....
\newline
-De aard van de markt kan zijn : volkomen concurrentie , monopolistische concurrentie , oligopolie , monopolie.
\newline
-Ook heeft de elasticiteit van de vraag een grote invloed op prijszetting.
\newline
-Andere factoren zoals sociaal , cultureel , wetgeving , herverkopers , .... kunnen ook een rol spelen
\end{antwoord}

\newpage
% =================================================================
\section{Module 2: Lineair Programmeren}
% =================================================================

\subsection{Definitie LP}
Wat is lineair programmeren? Wat zijn de belangrijkste elementen van een lineair programmeringsprobleem? Geef een definitie, beschrijf elk element en geef een praktisch voorbeeld.

\begin{antwoord}
% TYP HIERONDER JE ANTWOORD
Lineair programmeren is een wiskundige methode voor het bepalen van een doelfunctie (bvb : maximale winst of minimale kosten) bij lineaire relaties met beperkingen (bvb : grondstoffen , arbeidstijd).
\end{antwoord}


\subsection{Haalbaarheid vs. Optimaliteit}
Wat betekent het als een oplossing 'haalbaar' is? Wat is het verschil tussen een 'haalbare regio' en een 'optimale oplossing'?

\begin{antwoord}
% TYP HIERONDER JE ANTWOORD
De haalbare regio is het gebied waarin alle beperkingen van een lineair programmeringsprobleem worden nageleefd. Een oplossing is haalbaar als deze binnen deze regio ligt. De optimale oplossing is de specifieke punt binnen de haalbare regio die de doelfunctie maximaliseert of minimaliseert.
\end{antwoord}


\subsection{Simplexmethode Vereisten}
Wat zijn de vereisten om de simplexmethode toe te passen op een lineair programmeringsprobleem?

\begin{antwoord}
% TYP HIERONDER JE ANTWOORD
De vereisten voor de simplexmethode zijn:
\newline
1. Er moet een doelstelling zijn (bvb : maximaliseren of minimaliseren).
\newline
2. Er moeten beperkingen zijn.
\newline
3. Er moeten meerdere variabelen zijn.
\newline
4. Alle functies moeten lineair geschreven kunnen worden.
\end{antwoord}


\subsection{Basisoplossing}
Wat is een 'basisoplossing' in de simplexmethode? Hoe wordt deze bepaald?

\begin{antwoord}
% TYP HIERONDER JE ANTWOORD
De basisoplossing is een oplossing waarbij het aantal niet-nul variabelen gelijk is aan het aantal beperkingen. Deze wordt bepaald door een subset van variabelen te kiezen die de beperkingen precies voldoen, terwijl de overige variabelen op nul worden gezet.
\newline
Dit is eigelijk je starttableau in de simplexmethode.
\end{antwoord}


\subsection{Pivot-operatie}
Leg uit wat de 'pivot'-operatie is in de simplexmethode. Waarom is deze belangrijk?

\begin{antwoord}
% TYP HIERONDER JE ANTWOORD
Het proces waarbij een nieuwe variabele word toegevoegd aan de basis en een oude variabele wordt verwijderd,wordt een pivto genoemd. Dit is belangrijk omdat het de oplossing verbetert door de doelfunctie te optimaliseren.
\end{antwoord}


\subsection{Meerdere optimale oplossingen}
Hoe wordt een probleem met meerdere optimale oplossingen herkend in de simplexmethode?

\begin{antwoord}
% TYP HIERONDER JE ANTWOORD
Een probleem met meerdere optimale oplossingen wordt herkend wanneer er een nulcoëfficiënt is in de rij van de doelfunctie (deze rij wordt ook wel de 'cijferregel' genoemd) voor een niet-basisvariabele nadat een optimale oplossing is bereikt. Dit betekent dat er alternatieve oplossingen zijn die dezelfde optimale waarde opleveren.
\end{antwoord}


\subsection{Schaduwprijs (Shadow Price)}
Wat is een schaduwwaarde? Hoe wordt dit gebruikt in de interpretatie van de resultaten?

\begin{antwoord}
% TYP HIERONDER JE ANTWOORD
Een schaduwwaarde is de hoeveelheid extra winst die je krijgt als je één eenheid extra krijgt van iets wat beperkt is.
\end{antwoord}


\subsection{Sensitiviteitsanalyse}
Leg uit wat de sensitiviteitsanalyse inhoudt in de context van lineair programmeren.

\begin{antwoord}
% TYP HIERONDER JE ANTWOORD
Sensitiviteitsanalyse onderzoekt hoe de veranderingen in de parameters van een lineair programmeringsprobleem (zoals coëfficiënten van de doelfunctie of beperkingen) de optimale oplossing beïnvloeden. Het helpt bij het begrijpen van de robuustheid van de oplossing en bij het nemen van beslissingen onder onzekerheid.
\end{antwoord}


\subsection{Simplex vs. Grafisch}
Waarom is de simplexmethode efficiënter dan het grafisch oplossen van LP-problemen?

\begin{antwoord}
% TYP HIERONDER JE ANTWOORD
Het heeft geen noodzaak aan grafieken en is geschikter voor problemen met meer dan twee variabelen, terwijl grafische methoden beperkt zijn tot twee variabelen.
\end{antwoord}


\subsection{Beperkingen Simplex}
Wat zijn de beperkingen van de simplexmethode? Noem ten minste twee.

\begin{antwoord}
% TYP HIERONDER JE ANTWOORD
Beperkingen van de simplexmethode zijn onder andere:
\newline
1. Beperkt tot lineaire problemen.
\newline
2. Rekenintensief voor zeer grote problemen.
\end{antwoord}

\newpage
% =================================================================
\section{Module 3: Investeringsanalyse}
% =================================================================

\subsection{Cash Flow}
Wat verstaat men onder 'Cash Flow'? Leg uit.

\begin{antwoord}
% TYP HIERONDER JE ANTWOORD
Cash flow of netto kassastroom is het verschill tussen het geld dat binnenkomt en het geld dat uitgaat over een bepaalde periode.
\end{antwoord}


\subsection{Belastingen en Afschrijvingen}
Op welk bedrag worden de belastingen berekend? Leg uit en betrek het begrip 'Afschrijvingen' in je antwoord.

\begin{antwoord}
% TYP HIERONDER JE ANTWOORD
Belastingen worden berekend op de winst voor belastingen, afschrijvingen spelen hierbij een rol omdat ze deze totale winst verlagen (in de boekhouding) zodat er minder belastingen worden afgetrokken.
\end{antwoord}


\subsection{Actualiseringscoëfficiënt}
Wat is een 'Actualiserings-coëfficiënt'? Leg uit.

\begin{antwoord}
% TYP HIERONDER JE ANTWOORD
Sinds geld in de toekomst minder waard is dan geld nu door dingen zoals inflatie word een actualiseringscoëfficiënt gebruikt om toekomstige geldbedragen om te rekenen naar hun huidige waarde.
\end{antwoord}


\subsection{Afschrijvingen en Winstoptimalisatie}
Wat is het belang van 'Afschrijvingen' in functie van winstoptimalisatie? Leg uit.

\begin{antwoord}
% TYP HIERONDER JE ANTWOORD
Afschrijvingen verminderen de boekhoudkundige winst van een bedrijf, wat resulteert in lagere belastingen. Door afschrijvingen strategisch te plannen, kan een bedrijf zijn winst optimaliseren door de belastingdruk te verlagen en zo meer kapitaal beschikbaar te houden voor investeringen of andere doeleinden.
\end{antwoord}


\subsection{Afschrijvingsmethoden}
Bespreek twee manieren van afschrijven. Geef enkele voordelen en nadelen van de twee door jou besproken manieren. (Mag met voorbeeld).

\begin{antwoord}
% TYP HIERONDER JE ANTWOORD
(Er zijn 4 manieren van afschrijven : lineair,degressief,vertraag,reeële waarde)
\newline
-Lineaire afschrijving: Elke jaar hetzelfde bedrag afschrijven.
\newline
 Voordeel: Eenvoudig te berekenen en te begrijpen en word aanvaard door de fiscus(belastingsdienst).
\newline
 Nadeel: Je schrijft altijd een vast bedrag af, ongeacht het gebruik of de waarde van het actief.
\newline
-Degressieve afschrijving: In het begin meer afschrijven en later minder.
\newline
Voordeel: In het begin minder belastingen.
\newline
Nadeel: Word niet altijd aanvaard door de fiscus.
\end{antwoord}


\subsection{NHW en IRR}
Wat is de relatie tussen de 'Netto Huidige Waarde' (NHW) en de 'Internal Rate of Return' (IRR)? Leg uit.

\begin{antwoord}
% TYP HIERONDER JE ANTWOORD
De NHW is de som van alle toekomstige kassatromen min de intiële investering (en het houd rekening met de huidige waarde van geld van vandaag), indien deze groter is dan 0 dan is de investering rendabel. De IRR is het rendement dat je krijgt op een investering, dit kan dus bepaald worden door het NHW gelijk te stellen aan 0.
\end{antwoord}


\subsection{Schema Investeringsanalyse}
Hoe voer je een 'Investeringsanalyse' uit? Leg uit aan de hand van een schema.

\begin{antwoord}
% TYP HIERONDER JE ANTWOORD
1. Bepaal de initiële investering.
\newline
2. Bepaal de jaarlijkse kassatromen (inkomsten - uitgaven).
\newline
3. Bepaal de restwaarde aan het einde van de levensduur.
\newline
4. Bepaal de discontovoet (rentevoet).
\newline
5. Bereken de Netto Huidige Waarde (NHW) of Internal Rate of Return (IRR).
\newline
6. Maak een beslissing op basis van de NHW of IRR.
\end{antwoord}


\subsection{Casus: 10 jaar investering}
Stel je wilt een investering gaan analyseren die een 'effect' zou hebben over een periode van 10 jaar. Welke beoordelingscriteria zou je gaan gebruiken, en waarom? Motiveer duidelijk.

\begin{antwoord}
% TYP HIERONDER JE ANTWOORD
Een discounted flow methode zoals NHW of IRR omdat deze rekening houden met de tijdswaarde van geld over een langere periode zoals 10 jaar.
\end{antwoord}


\subsection{Inflatie}
Definieer duidelijk het begrip 'inflatie'. Welke impact heeft inflatie op investeringen? Leg uit.

\begin{antwoord}
% TYP HIERONDER JE ANTWOORD
Inflatie is de vermindering van koopkracht van een vaste hoeveelheid geld over een bepaalde periode. Dit leidt tot een lagere netto winst waarde van een investering. Soms kan het echter een positief effect hebben als inflatie zorgt dat materialen in prijs stijgen en dus de investering in iets (bvb : een machine) rendabeler maakt.
\end{antwoord}


\subsection{Subsidies}
Investeringen en subsidies. Hoe kunnen subsidies bepalend zijn bij een investeringsevaluatie? Leg uit.

\begin{antwoord}
% TYP HIERONDER JE ANTWOORD
Als je subsidies krijgt voor een investering, verlaagt dit de initiële kosten van de investering, wat de netto huidige waarde (NHW) kan verhogen en de internal rate of return (IRR) kan verbeteren. Hierdoor kan een investering die anders niet rendabel zou zijn, toch aantrekkelijk worden.
\end{antwoord}

\newpage
% =================================================================
\section{Module 4: Kostprijscalculatie}
% =================================================================

\subsection{Soorten Prijzen}
Hoe verhouden 'Verkoopprijs', 'Totale Kostprijs', 'Industriële Kostprijs' en 'Evenredige Kostprijs' zich tot elkaar? Leg uit met nuance.

\begin{antwoord}
% TYP HIERONDER JE ANTWOORD
De verkoopprijs is de prijs waarvoor een product aan klanten wordt verkocht. De totale kostprijs bevat alle kosten die gemaakt worden om het product te MAKEN en te VERKOPEN, dus zonder winst. De Industriële kostprijs omvat alleen kosten die direct gerelateerd zijn aan productie. De evenredige kostprijs is de kost van alle variabele kosten per eenheid, deze word gebruikt om de marge te bepalen.
\end{antwoord}


\subsection{Historische vs. Standaardkostprijzen}
We spreken over 'Historische Kostprijzen' en 'Standaardkostprijzen'. Wat is het verschil en wat is het belang ervan?

\begin{antwoord}
% TYP HIERONDER JE ANTWOORD
Historische kostprijzen zijn gebaseerd op werkelijke kosten die in het verleden zijn gemaakt, terwijl standaardkostprijzen vooraf bepaalde kosten zijn die worden gebruikt voor budgettering en kostenbeheersing. Het belang van standaardkostprijzen ligt in het vergemakkelijken van kostencontrole en prestatie-evaluatie, terwijl historische kostprijzen nuttig zijn voor het analyseren van werkelijke prestaties (een benchmark).
\end{antwoord}


\subsection{Verdeelsleutel}
Wat wordt bedoeld met een 'Verdeelsleutel' in het kader van kostprijscalculatie? Leg uit (aan de hand van een voorbeeld).

\begin{antwoord}
% TYP HIERONDER JE ANTWOORD
Een verdeelsleutel word gebruikt om de indirecte kosten als toeslag toe te berekenen aan de kostprijs van een product. Verdeelsleutel bepaald welke sectoren of afdelingen welke kosten moeten dragen. Bijvoorbeeld, als een fabriek 1000 euro aan indirecte kosten heeft en er 500 geproduceerde eenheden zijn, dan is de verdeelsleutel 2 euro per eenheid (1000/500).
\end{antwoord}


\subsection{Directe en Indirecte Kosten}
Wat zijn 'Directe Kosten' en 'Indirecte Kosten'? Wat is het belang ervan bij kostprijscalculatie? Wat zijn de problemen? Leg uit.

\begin{antwoord}
% TYP HIERONDER JE ANTWOORD
Directe kosten zijn kosten die gelinkt worden aan de productie van een specifiek product (zoals grondstoffen en directe arbeid). Indirecte kosten zijn kosten die niet direct aan een specifiek product kunnen worden toegeschreven (zoals huur, elektriciteit, administratiekosten). Het belang van het onderscheiden van deze kosten ligt in het nauwkeurig berekenen van de kostprijs van een product. Problemen kunnen ontstaan bij het toewijzen van indirecte kosten, omdat dit subjectief kan zijn en kan leiden tot onnauwkeurige kostprijzen.
\end{antwoord}


\subsection{Vaste en Variabele Kosten}
Wat zijn 'Vaste Kosten' en 'Variabele Kosten'? Wat is het belang ervan bij kostprijscalculatie? Leg uit.

\begin{antwoord}
% TYP HIERONDER JE ANTWOORD
Vaste kosten staan vast, veranderen niet met de productiehoeveelheid (zoals huur, salarissen). Variabele kosten veranderen wel met de productiehoeveelheid (zoals grondstoffen, directe arbeid). Het belang van het onderscheiden van deze kosten ligt in het begrijpen van hoe kosten zich gedragen bij veranderingen in productievolumes, wat essentieel is voor kostprijscalculatie en winstoptimalisatie.
\end{antwoord}


\subsection{Kostprijs vs. Winstgevendheid}
Kostprijscalculatie en winstgevendheid van een bepaald product in het assortiment. Is er een relatie? Leg uit (evt. met voorbeeld).

\begin{antwoord}
% TYP HIERONDER JE ANTWOORD
Ja, er is een relatie. Kostprijscalculatie helpt bij het bepalen van de kostprijs van een product, wat essentieel is voor het vaststellen van de verkoopprijs. Als de verkoopprijs hoger is dan de kostprijs, is het product winstgevend. Bijvoorbeeld, als de kostprijs van een product 50 euro is en het wordt verkocht voor 70 euro, dan is de winst per eenheid 20 euro.
\end{antwoord}


\subsection{Break-Even Point}
Leg uit: 'Break-Even Point'.

\begin{antwoord}
% TYP HIERONDER JE ANTWOORD
Het punt waarbij je net geen winst of verlies maakt. Dit is het punt waarop de totale opbrengsten gelijk zijn aan de totale kosten.
\end{antwoord}


\subsection{ABC Costing}
Wat wordt bedoeld met 'ABC costing'?
\begin{itemize}
    \item Wat zijn de fundamentele verschillen met traditionele methoden?
    \item Wat zijn de voor- en nadelen?
\end{itemize}

\begin{antwoord}
% TYP HIERONDER JE ANTWOORD
ABC (Activity Based Costing) is een elegantere methode voor het verdelen van de kosten over producten of diensten. In tegenstelling tot traditionele methoden, die vaak een enkele verdeelsleutel gebruiken, identificeert ABC de verschillende activiteiten die kosten veroorzaken en wijst deze kosten toe op basis van het daadwerkelijke gebruik van die activiteiten door producten of diensten. Deze methode van kostenverdeling is wel complexer maar biedt een eerlijker zicht in wat de kostprijs van een product of dienst is.
\end{antwoord}


\subsection{Afwijkingsanalyse}
Wat wordt bedoeld met een 'Afwijkingsanalyse' in het kader van kostprijscalculatie? Wat is het nut ervan?

\begin{antwoord}
% TYP HIERONDER JE ANTWOORD
IS DIT TE KENNEN?
\newline
Afwijkingsanalyse is het proces van het vergelijken van werkelijke kosten met de standaard- of budgetkosten om verschillen (afwijkingen) te identificeren. Het nut ervan ligt in het helpen van bedrijven om inefficiënties te identificeren, kostenbeheersing te verbeteren en betere beslissingen te nemen op basis van kostenprestaties.
\end{antwoord}


\subsection{Target Costing}
Wat is het verschil tussen 'Traditionele Kostprijscalculatie' en 'Target Costing'? Leg uit (mag met schema).

\begin{antwoord}
% TYP HIERONDER JE ANTWOORD
Bij traditionele kostprijscalculatie word eerst het proces van het maken van het product gedaan en daarna de kosten hiervan bepaalt. Bij Target Costing word eerst de verkoopprijs bepaald op basis van markt- en klantbehoeften, daarna word de gewenste winst afgetrokken om zo de target kostprijs te bepalen. Vervolgens word het product ontworpen en geproduceerd binnen deze kostprijs.
\end{antwoord}

\newpage
% =================================================================
\section{Module 5: Algemene Boekhouding en Financiële Analyse}
% =================================================================

\subsection{Ratio-analyses: Nut en Beperking}
Waarom worden 'Ratio-analyses' gebruikt binnen het Financieel Management? Wat zijn de beperkingen?

\begin{antwoord}
% TYP HIERONDER JE ANTWOORD
Ratio-analyses worden gebruikt om de financiële gezondheid en prestaties van een bedrijf te beoordelen door verschillende financiële indicatoren te vergelijken. Ze helpen bij het identificeren van trends, het vergelijken met concurrenten en het nemen van geïnformeerde beslissingen. Beperkingen zijn onder andere dat ze afhankelijk zijn van historische gegevens, kunnen worden beïnvloed door boekhoudkundige methoden en niet altijd een volledig beeld geven van de operationele efficiëntie of marktomstandigheden.
\end{antwoord}


\subsection{Balance Sheet}
Wat is een 'Balance Sheet' (Balans)? Leg uit.

\begin{antwoord}
% TYP HIERONDER JE ANTWOORD
Het is een overzicht van de financiële situatie van een bedrijf op een bepaald moment. Het toont wat het bedrijf bezit (activa), wat het verschuldigd is (passiva) en het eigen vermogen van de aandeelhouders.
\end{antwoord}


\subsection{Balans vs. Resultatenrekening}
Wat is het essentiële verschil tussen de 'Balans' en de 'Resultatenrekening'?

\begin{antwoord}
% TYP HIERONDER JE ANTWOORD
De balans geeft een momentopname van de financiële positie van een bedrijf op een bepaald tijdstip, terwijl de resultatenrekening (of winst-en-verliesrekening) de financiële prestaties van het bedrijf over een bepaalde periode weergeeft, inclusief inkomsten en uitgaven.
\end{antwoord}


\subsection{De Jaarrekening}
Bespreek de 'Jaarrekening'. Wat zijn de essentiële elementen?

\begin{antwoord}
% TYP HIERONDER JE ANTWOORD
De jaarrekening bestaat uit drie essentiële elementen: de balans, de resultatenrekening en het kasstroomoverzicht. De balans toont de activa, passiva en het eigen vermogen van het bedrijf op een bepaald moment. De resultatenrekening geeft inzicht in de inkomsten, kosten en winst of verlies over een bepaalde periode. Het kasstroomoverzicht laat zien hoe geld binnenkomt en uitgaat, wat belangrijk is voor het beoordelen van de liquiditeit van het bedrijf.
\end{antwoord}


\subsection{Horizontale vs. Verticale Analyse}
Wat is het verschil tussen een 'Horizontale' en 'Verticale' analyse bij financiële analyses?

\begin{antwoord}
% TYP HIERONDER JE ANTWOORD
Horizontale analyse vergelijkt financiële gegevens over meerdere perioden om trends en veranderingen in prestaties te identificeren. Verticale analyse daarentegen bekijkt de verhouding van elk item op een financiële verklaring ten opzichte van een basisitem binnen dezelfde periode, waardoor de structuur van de financiële verklaring wordt geanalyseerd.
\end{antwoord}


\subsection{Ratio's}
Geef 4 ratio's en bespreek deze. (Je mag vrij kiezen dewelke).

\begin{antwoord}
% TYP HIERONDER JE ANTWOORD
1. (Liquiditeitsratio) Current Ratio: Dit meet de liquiditeit van een bedrijf door de verhouding tussen vlottende activa en vlottende passiva te berekenen. Een ratio boven 1 duidt op voldoende kortetermijnmiddelen om verplichtingen te dekken.
\newline
2. (Solvabiliteitsratio) Debt ratio: Dit meet de financiële hefboomwerking door de verhouding tussen totale schulden en totale activa te berekenen. Een hogere ratio kan wijzen op een hoger risico voor schuldeisers.
\newline
3. Schuldenratio: Dit meet de verhouding tussen vreemd vermogen en eigen vermogen. Het geeft inzicht in de financiële structuur van het bedrijf en het risico voor aandeelhouders.
\newline
4. Koers-winst ratio (K/W ratio): Dit meet de waardering van een aandeel door de verhouding tussen de marktprijs per aandeel en de winst per aandeel te berekenen. Een hogere K/W ratio kan wijzen op hoge verwachtingen van toekomstige groei.
\end{antwoord}

\end{document}