\documentclass[a4paper,11pt]{article}
\usepackage[dutch]{babel}
\usepackage[utf8]{inputenc}
\usepackage{amsmath}
\usepackage{geometry}
\usepackage{parskip}
\usepackage{graphicx}
\usepackage{xcolor} % Nodig voor kleuren

% Marges instellen
\geometry{a4paper, total={170mm,257mm}, left=20mm, top=20mm}

% --- DEFINITIE VAN HET ANTWOORD-BLOK ---
% Alles tussen \begin{antwoord} en \end{antwoord} wordt blauw.
\newenvironment{antwoord}
    {
    \vspace{0.1cm}      % Beetje witruimte boven
    \color{blue}        % Tekstkleur blauw
    \textbf{} \\ % Labeltje
    }
    {
    \vspace{0.2cm}      % Witruimte onder
    \hrule              % Horizontale lijn
    \vspace{0.3cm}      % Witruimte na de lijn
    \color{black}       % Terug naar zwart (veiligheid)
    }
% ---------------------------------------

\title{\textbf{Volledige Uitwerking Examenvragen}\\ \large Ingenieur en Economie (Versie 2025-2026)}
\author{Louic Cooreman}
\date{\today}

\begin{document}

\maketitle
\tableofcontents
\newpage

% =================================================================
\section{Module 1: Fundamenten van de Marketing}
% =================================================================

\subsection{Product Life Cycle (PLC) Figuur}
De onderstaande figuur stelt de PLC voor. Vul de fasen (1 t/m 5) aan en zet de assen erbij.

\begin{antwoord}
% TYP HIERONDER JE ANTWOORD
Fase 1: ...
Fase 2: ...
\end{antwoord}


\subsection{Marketingdoelstellingen en Prijszetting in de PLC}
Bespreek kort de marketingdoelstelling en de prijszetting in elke fase van de PLC.

\begin{antwoord}
% TYP HIERONDER JE ANTWOORD
Typ hier je antwoord...
\end{antwoord}


\subsection{Elasticiteit van de vraag}
Wat is elasticiteit van de vraag? Leg uit aan de hand van een schets en een voorbeeld.

\begin{antwoord}
% TYP HIERONDER JE ANTWOORD
Typ hier je antwoord...
\end{antwoord}


\subsection{De 5 Marketingconcepten}
Geef de 5 marketingconcepten. Bespreek elk concept kort en geef een voorbeeld van elk.

\begin{antwoord}
% TYP HIERONDER JE ANTWOORD
Typ hier je antwoord...
\end{antwoord}


\subsection{Het Marketingmanagementconcept}
Definieer 'marketingmanagementconcept'. Wat is het belang ervan voor consument en organisatie? Bespreek de 5 belangrijkste en geef duidelijk de verschilpunten aan.

\begin{antwoord}
% TYP HIERONDER JE ANTWOORD
Typ hier je antwoord...
\end{antwoord}


\subsection{Segmentatie en Clustering}
\begin{itemize}
    \item Wat zijn segmentatievariabelen? Bespreek de belangrijkste categorieën en geef een voorbeeld van elk.
    \item Wat wordt bedoeld met 'clustering van doelgroepen'? Leg uit.
\end{itemize}

\begin{antwoord}
% TYP HIERONDER JE ANTWOORD
Typ hier je antwoord...
\end{antwoord}


\subsection{Prijsstrategieën}
Wat wordt bedoeld met een \textbf{penetratieprijsstrategie} en een \textbf{afroomprijsstrategie}? Wanneer pas je ze toe?

\begin{antwoord}
% TYP HIERONDER JE ANTWOORD
Typ hier je antwoord...
\end{antwoord}


\subsection{Prijsvork}
Wat wordt bedoeld met een 'prijsvork'?

\begin{antwoord}
% TYP HIERONDER JE ANTWOORD
Typ hier je antwoord...
\end{antwoord}


\subsection{Integratievraag}
Ondernemerschap, Maatschappelijke ontwikkeling, Investeringen, Productontwikkeling, Toevoeging van waarde, Kostprijscalculatie, Gezonde bedrijven en welvaart... Hoe hangt het allemaal aan elkaar? Bespreek.

\begin{antwoord}
% TYP HIERONDER JE ANTWOORD
Typ hier je antwoord...
\end{antwoord}


\subsection{De Marketingmix (4P's)}
Bespreek de 4P's in het kader van de 'Marketingmix'.

\begin{antwoord}
% TYP HIERONDER JE ANTWOORD
Typ hier je antwoord...
\end{antwoord}


\subsection{Prijszetting factoren}
Welke factoren zijn van belang voor de prijszetting? Splits ze op naar 'interne' en 'externe' factoren.

\begin{antwoord}
% TYP HIERONDER JE ANTWOORD
Typ hier je antwoord...
\end{antwoord}

\newpage
% =================================================================
\section{Module 2: Lineair Programmeren}
% =================================================================

\subsection{Definitie LP}
Wat is lineair programmeren? Wat zijn de belangrijkste elementen van een lineair programmeringsprobleem? Geef een definitie, beschrijf elk element en geef een praktisch voorbeeld.

\begin{antwoord}
% TYP HIERONDER JE ANTWOORD
Lineair programmeren is een wiskundige methode voor het bepalen van een doelfunctie (bvb : maximale winst of minimale kosten) bij lineaire relaties met beperkingen (bvb : grondstoffen , arbeidstijd).
\end{antwoord}


\subsection{Haalbaarheid vs. Optimaliteit}
Wat betekent het als een oplossing 'haalbaar' is? Wat is het verschil tussen een 'haalbare regio' en een 'optimale oplossing'?

\begin{antwoord}
% TYP HIERONDER JE ANTWOORD
De haalbare regio is het gebied waarin alle beperkingen van een lineair programmeringsprobleem worden nageleefd. Een oplossing is haalbaar als deze binnen deze regio ligt. De optimale oplossing is de specifieke punt binnen de haalbare regio die de doelfunctie maximaliseert of minimaliseert.
\end{antwoord}


\subsection{Simplexmethode Vereisten}
Wat zijn de vereisten om de simplexmethode toe te passen op een lineair programmeringsprobleem?

\begin{antwoord}
% TYP HIERONDER JE ANTWOORD
De vereisten voor de simplexmethode zijn:
\newline
1. Er moet een doelstelling zijn (bvb : maximaliseren of minimaliseren).
\newline
2. Er moeten beperkingen zijn.
\newline
3. Er moeten meerdere variabelen zijn.
\newline
4. Alle functies moeten lineair geschreven kunnen worden.
\end{antwoord}


\subsection{Basisoplossing}
Wat is een 'basisoplossing' in de simplexmethode? Hoe wordt deze bepaald?

\begin{antwoord}
% TYP HIERONDER JE ANTWOORD
De basisoplossing is een oplossing waarbij het aantal niet-nul variabelen gelijk is aan het aantal beperkingen. Deze wordt bepaald door een subset van variabelen te kiezen die de beperkingen precies voldoen, terwijl de overige variabelen op nul worden gezet.
\newline
Dit is eigelijk je starttableau in de simplexmethode.
\end{antwoord}


\subsection{Pivot-operatie}
Leg uit wat de 'pivot'-operatie is in de simplexmethode. Waarom is deze belangrijk?

\begin{antwoord}
% TYP HIERONDER JE ANTWOORD
Het proces waarbij een nieuwe variabele word toegevoegd aan de basis en een oude variabele wordt verwijderd,wordt een pivto genoemd. Dit is belangrijk omdat het de oplossing verbetert door de doelfunctie te optimaliseren.
\end{antwoord}


\subsection{Meerdere optimale oplossingen}
Hoe wordt een probleem met meerdere optimale oplossingen herkend in de simplexmethode?

\begin{antwoord}
% TYP HIERONDER JE ANTWOORD
Een probleem met meerdere optimale oplossingen wordt herkend wanneer er een nulcoëfficiënt is in de rij van de doelfunctie (deze rij wordt ook wel de 'cijferregel' genoemd) voor een niet-basisvariabele nadat een optimale oplossing is bereikt. Dit betekent dat er alternatieve oplossingen zijn die dezelfde optimale waarde opleveren.
\end{antwoord}


\subsection{Schaduwprijs (Shadow Price)}
Wat is een schaduwwaarde? Hoe wordt dit gebruikt in de interpretatie van de resultaten?

\begin{antwoord}
% TYP HIERONDER JE ANTWOORD
Een schaduwwaarde is de hoeveelheid extra winst die je krijgt als je één eenheid extra krijgt van iets wat beperkt is.
\end{antwoord}


\subsection{Sensitiviteitsanalyse}
Leg uit wat de sensitiviteitsanalyse inhoudt in de context van lineair programmeren.

\begin{antwoord}
% TYP HIERONDER JE ANTWOORD
Sensitiviteitsanalyse onderzoekt hoe de veranderingen in de parameters van een lineair programmeringsprobleem (zoals coëfficiënten van de doelfunctie of beperkingen) de optimale oplossing beïnvloeden. Het helpt bij het begrijpen van de robuustheid van de oplossing en bij het nemen van beslissingen onder onzekerheid.
\end{antwoord}


\subsection{Simplex vs. Grafisch}
Waarom is de simplexmethode efficiënter dan het grafisch oplossen van LP-problemen?

\begin{antwoord}
% TYP HIERONDER JE ANTWOORD
Het heeft geen noodzaak aan grafieken en is geschikter voor problemen met meer dan twee variabelen, terwijl grafische methoden beperkt zijn tot twee variabelen.
\end{antwoord}


\subsection{Beperkingen Simplex}
Wat zijn de beperkingen van de simplexmethode? Noem ten minste twee.

\begin{antwoord}
% TYP HIERONDER JE ANTWOORD
Beperkingen van de simplexmethode zijn onder andere:
\newline
1. Beperkt tot lineaire problemen.
\newline
2. Rekenintensief voor zeer grote problemen.
\end{antwoord}

\newpage
% =================================================================
\section{Module 3: Investeringsanalyse}
% =================================================================

\subsection{Cash Flow}
Wat verstaat men onder 'Cash Flow'? Leg uit.

\begin{antwoord}
% TYP HIERONDER JE ANTWOORD
Cash flow of netto kassastroom is het verschill tussen het geld dat binnenkomt en het geld dat uitgaat over een bepaalde periode.
\end{antwoord}


\subsection{Belastingen en Afschrijvingen}
Op welk bedrag worden de belastingen berekend? Leg uit en betrek het begrip 'Afschrijvingen' in je antwoord.

\begin{antwoord}
% TYP HIERONDER JE ANTWOORD
Belastingen worden berekend op de winst voor belastingen, afschrijvingen spelen hierbij een rol omdat ze deze totale winst verlagen (in de boekhouding) zodat er minder belastingen worden afgetrokken.
\end{antwoord}


\subsection{Actualiseringscoëfficiënt}
Wat is een 'Actualiserings-coëfficiënt'? Leg uit.

\begin{antwoord}
% TYP HIERONDER JE ANTWOORD
Sinds geld in de toekomst minder waard is dan geld nu door dingen zoals inflatie word een actualiseringscoëfficiënt gebruikt om toekomstige geldbedragen om te rekenen naar hun huidige waarde.
\end{antwoord}


\subsection{Afschrijvingen en Winstoptimalisatie}
Wat is het belang van 'Afschrijvingen' in functie van winstoptimalisatie? Leg uit.

\begin{antwoord}
% TYP HIERONDER JE ANTWOORD
Typ hier je antwoord...
\end{antwoord}


\subsection{Afschrijvingsmethoden}
Bespreek twee manieren van afschrijven. Geef enkele voordelen en nadelen van de twee door jou besproken manieren. (Mag met voorbeeld).

\begin{antwoord}
% TYP HIERONDER JE ANTWOORD
(Er zijn 4 manieren van afschrijven : lineair,degressief,vertraag,reeële waarde)
\newline
-Lineaire afschrijving: Elke jaar hetzelfde bedrag afschrijven.
\newline
 Voordeel: Eenvoudig te berekenen en te begrijpen en word aanvaard door de fiscus(belastingsdienst).
\newline
 Nadeel: Je schrijft altijd een vast bedrag af, ongeacht het gebruik of de waarde van het actief.
\newline
-Degressieve afschrijving: In het begin meer afschrijven en later minder.
\newline
Voordeel: In het begin minder belastingen.
\newline
Nadeel: Word niet altijd aanvaard door de fiscus.
\end{antwoord}


\subsection{NHW en IRR}
Wat is de relatie tussen de 'Netto Huidige Waarde' (NHW) en de 'Internal Rate of Return' (IRR)? Leg uit.

\begin{antwoord}
% TYP HIERONDER JE ANTWOORD
De NHW is de som van alle toekomstige kassatromen min de intiële investering (en het houd rekening met de huidige waarde van geld van vandaag), indien deze groter is dan 0 dan is de investering rendabel. De IRR is het rendement dat je krijgt op een investering, dit kan dus bepaald worden door het NHW gelijk te stellen aan 0.
\end{antwoord}


\subsection{Schema Investeringsanalyse}
Hoe voer je een 'Investeringsanalyse' uit? Leg uit aan de hand van een schema.

\begin{antwoord}
% TYP HIERONDER JE ANTWOORD
Typ hier je antwoord...
\end{antwoord}


\subsection{Casus: 10 jaar investering}
Stel je wilt een investering gaan analyseren die een 'effect' zou hebben over een periode van 10 jaar. Welke beoordelingscriteria zou je gaan gebruiken, en waarom? Motiveer duidelijk.

\begin{antwoord}
% TYP HIERONDER JE ANTWOORD
Een discounted flow methode zoals NHW of IRR omdat deze rekening houden met de tijdswaarde van geld over een langere periode zoals 10 jaar.
\end{antwoord}


\subsection{Inflatie}
Definieer duidelijk het begrip 'inflatie'. Welke impact heeft inflatie op investeringen? Leg uit.

\begin{antwoord}
% TYP HIERONDER JE ANTWOORD
Inflatie is de vermindering van koopkracht van een vaste hoeveelheid geld over een bepaalde periode. Dit leidt tot een lagere netto winst waarde van een investering. Soms kan het echter een positief effect hebben als inflatie zorgt dat materialen in prijs stijgen en dus de investering in iets (bvb : een machine) rendabeler maakt.
\end{antwoord}


\subsection{Subsidies}
Investeringen en subsidies. Hoe kunnen subsidies bepalend zijn bij een investeringsevaluatie? Leg uit.

\begin{antwoord}
% TYP HIERONDER JE ANTWOORD
Als je subsidies krijgt voor een investering, verlaagt dit de initiële kosten van de investering, wat de netto huidige waarde (NHW) kan verhogen en de internal rate of return (IRR) kan verbeteren. Hierdoor kan een investering die anders niet rendabel zou zijn, toch aantrekkelijk worden.
\end{antwoord}

\newpage
% =================================================================
\section{Module 4: Kostprijscalculatie}
% =================================================================

\subsection{Soorten Prijzen}
Hoe verhouden 'Verkoopprijs', 'Totale Kostprijs', 'Industriële Kostprijs' en 'Evenredige Kostprijs' zich tot elkaar? Leg uit met nuance.

\begin{antwoord}
% TYP HIERONDER JE ANTWOORD
Typ hier je antwoord...
\end{antwoord}


\subsection{Historische vs. Standaardkostprijzen}
We spreken over 'Historische Kostprijzen' en 'Standaardkostprijzen'. Wat is het verschil en wat is het belang ervan?

\begin{antwoord}
% TYP HIERONDER JE ANTWOORD
Typ hier je antwoord...
\end{antwoord}


\subsection{Verdeelsleutel}
Wat wordt bedoeld met een 'Verdeelsleutel' in het kader van kostprijscalculatie? Leg uit (aan de hand van een voorbeeld).

\begin{antwoord}
% TYP HIERONDER JE ANTWOORD
Typ hier je antwoord...
\end{antwoord}


\subsection{Directe en Indirecte Kosten}
Wat zijn 'Directe Kosten' en 'Indirecte Kosten'? Wat is het belang ervan bij kostprijscalculatie? Wat zijn de problemen? Leg uit.

\begin{antwoord}
% TYP HIERONDER JE ANTWOORD
Typ hier je antwoord...
\end{antwoord}


\subsection{Vaste en Variabele Kosten}
Wat zijn 'Vaste Kosten' en 'Variabele Kosten'? Wat is het belang ervan bij kostprijscalculatie? Leg uit.

\begin{antwoord}
% TYP HIERONDER JE ANTWOORD
Typ hier je antwoord...
\end{antwoord}


\subsection{Kostprijs vs. Winstgevendheid}
Kostprijscalculatie en winstgevendheid van een bepaald product in het assortiment. Is er een relatie? Leg uit (evt. met voorbeeld).

\begin{antwoord}
% TYP HIERONDER JE ANTWOORD
Typ hier je antwoord...
\end{antwoord}


\subsection{Break-Even Point}
Leg uit: 'Break-Even Point'.

\begin{antwoord}
% TYP HIERONDER JE ANTWOORD
Typ hier je antwoord...
\end{antwoord}


\subsection{ABC Costing}
Wat wordt bedoeld met 'ABC costing'?
\begin{itemize}
    \item Wat zijn de fundamentele verschillen met traditionele methoden?
    \item Wat zijn de voor- en nadelen?
\end{itemize}

\begin{antwoord}
% TYP HIERONDER JE ANTWOORD
Typ hier je antwoord...
\end{antwoord}


\subsection{Afwijkingsanalyse}
Wat wordt bedoeld met een 'Afwijkingsanalyse' in het kader van kostprijscalculatie? Wat is het nut ervan?

\begin{antwoord}
% TYP HIERONDER JE ANTWOORD
Typ hier je antwoord...
\end{antwoord}


\subsection{Target Costing}
Wat is het verschil tussen 'Traditionele Kostprijscalculatie' en 'Target Costing'? Leg uit (mag met schema).

\begin{antwoord}
% TYP HIERONDER JE ANTWOORD
Typ hier je antwoord...
\end{antwoord}

\newpage
% =================================================================
\section{Module 5: Algemene Boekhouding en Financiële Analyse}
% =================================================================

\subsection{Ratio-analyses: Nut en Beperking}
Waarom worden 'Ratio-analyses' gebruikt binnen het Financieel Management? Wat zijn de beperkingen?

\begin{antwoord}
% TYP HIERONDER JE ANTWOORD
Typ hier je antwoord...
\end{antwoord}


\subsection{Balance Sheet}
Wat is een 'Balance Sheet' (Balans)? Leg uit.

\begin{antwoord}
% TYP HIERONDER JE ANTWOORD
Typ hier je antwoord...
\end{antwoord}


\subsection{Balans vs. Resultatenrekening}
Wat is het essentiële verschil tussen de 'Balans' en de 'Resultatenrekening'?

\begin{antwoord}
% TYP HIERONDER JE ANTWOORD
Typ hier je antwoord...
\end{antwoord}


\subsection{De Jaarrekening}
Bespreek de 'Jaarrekening'. Wat zijn de essentiële elementen?

\begin{antwoord}
% TYP HIERONDER JE ANTWOORD
Typ hier je antwoord...
\end{antwoord}


\subsection{Horizontale vs. Verticale Analyse}
Wat is het verschil tussen een 'Horizontale' en 'Verticale' analyse bij financiële analyses?

\begin{antwoord}
% TYP HIERONDER JE ANTWOORD
Typ hier je antwoord...
\end{antwoord}


\subsection{Ratio's}
Geef 4 ratio's en bespreek deze. (Je mag vrij kiezen dewelke).

\begin{antwoord}
% TYP HIERONDER JE ANTWOORD
Typ hier je antwoord...
\end{antwoord}

\end{document}